约数定理:若$n=\prod_{i=1}^kp_i^{a_i}$,则

1.约数个数$f(n)=\prod_{i=1}^k(a_i+1)$

2.约数和$g(n)=\prod_{i=1}^k(\sum_{j=0}^{a_i}p_i^j)$

小于$n$且互素的数之和为$n\varphi(n)/2$

若$gcd(n,i)=1$,则$gcd(n,n-i)=1(1\leq i\leq n)$

错排公式:$D(n)=(n-1)(D(n-2)+D(n-1))=\sum_{i=2}^n\frac{(-1)^kn!}{k!}=[\frac{n!}{e}+0.5]$

威尔逊定理:$p\ is\ prime\ \Rightarrow (p-1)!\equiv-1(mod\ p)$

欧拉定理:$gcd(a,n)=1\Rightarrow a^{\varphi(n)}\equiv1(mod\ n)$

欧拉定理推广:$gcd(n,p)=1\Rightarrow a^n\equiv a^{n\%\varphi(p)}(mod\ p)$

素数定理:对于不大于n的素数个数$\pi(n)$,$\lim\limits_{n\to\infty}\pi(n)=\frac{n}{\ln n}$

位数公式:正整数$x$的位数$N=log10(n)+1$

斯特灵公式$n!\approx\sqrt{2\pi n}(\frac{n}{e})^n$

设$a>1,m,n>0$,则$gcd(a^m-1,a^n-1)=a^{gcd(m,n)}-1$

设$a>b,gcd(a,b)=1$,则$gcd(a^m-b^m,a^n-b^n)=a^{gcd(m,n)}-b^{gcd(m,n)}$

$$
	G=gcd(C_n^1,C_n^2,...,C_n^{n-1})=
	\begin{cases}
		n, & \text{$n$ is prime}                    \\
		1, & \text{$n$ has multy prime factors}     \\
		p, & \text{$n$ has single prime factor $p$}
	\end{cases}
$$

$gcd(Fib(m),Fib(n))=Fib(gcd(m,n))$

若$gcd(m,n)=1$,则:

1.最大不能组合的数为$m*n-m-n$

2.不能组合数个数$N=\frac{(m-1)(n-1)}{2}$

$(n+1)lcm(C_n^0,C_n^1,...,C_n^{n-1},C_n^{n})=lcm(1,2,...,n+1)$

若$p$为素数,则$(x+y+...+w)^p\equiv x^p+y^p+...+w^p(mod\ p)$

卡特兰数:1, 1, 2, 5, 14, 42, 132, 429, 1430, 4862, 16796, 58786, 208012

$h(0)=h(1)=1,h(n)=\frac{(4n-2)h(n-1)}{n+1}=\frac{C_{2n}^n}{n+1}=C_{2n}^n-C_{2n}^{n-1}$

$$
	a_{n+m}=\sum_{i=0}^{m-1}b_ia_{n+i}\Rightarrow
	\left(
	\begin{matrix}
			a_{n+m}   \\\\
			a_{n+m-1} \\\\
			\vdots    \\\\
			a_{n+1}   \\\\
		\end{matrix}
	\right)
	=
	\left(
	\begin{matrix}
			b_{m-1} &  & \cdots &  & b_1    &  & b_0    \\\\
			1       &  & \cdots &  & 0      &  & 0      \\\\
			\vdots  &  & \ddots &  & \vdots &  & \vdots \\\\
			0       &  & \cdots &  & 1      &  & 0      \\\\
		\end{matrix}
	\right)
	\left(
	\begin{matrix}
			a_{n+m-1} \\\\
			a_{n+m-2} \\\\
			\vdots    \\\\
			a_n       \\\\
		\end{matrix}
	\right)
$$

$$
	a_{n+m}=\sum_{i=0}^{m-1}b_ia_{n+i}+c\Rightarrow
	\left(
	\begin{matrix}
			a_{n+m}   \\\\
			a_{n+m-1} \\\\
			\vdots    \\\\
			a_{n+1}   \\\\
			1         \\\\
		\end{matrix}
	\right)
	=
	\left(
	\begin{matrix}
			b_{m-1} &  & \cdots &  & b_1    &  & b_0    &  & c      \\\\
			1       &  & \cdots &  & 0      &  & 0      &  & 0      \\\\
			\vdots  &  & \ddots &  & \vdots &  & \vdots &  & \vdots \\\\
			0       &  & \cdots &  & 1      &  & 0      &  & 0      \\\\
			0       &  & \cdots &  & 0      &  & 0      &  & 1      \\\\
		\end{matrix}
	\right)
	\left(
	\begin{matrix}
			a_{n+m-1} \\\\
			a_{n+m-2} \\\\
			\vdots    \\\\
			a_n       \\\\
			1         \\\\
		\end{matrix}
	\right)
$$